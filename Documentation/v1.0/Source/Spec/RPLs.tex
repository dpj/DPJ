\section{Region Path Lists}
\label{section:rpls}

In DPJ, the programmer uses region path lists, or RPLs, to name sets
of heap regions (\S~\ref{section:regions:heap}).  The region sets in
turn are used in specifying and checking effects
(\S~\ref{section:effects}).  \S~\ref{section:rpls:valid} and
\S~\ref{section:rpls:elts} describe the construction of RPLs.
\S~\ref{section:rpls:relations} describe the rules for comparing RPLs,
which are essential in noninterference checking.
\S~\ref{section:rpls:regions} describes \emph{heap regions} and
\emph{stack regions}, which do not appear in DPJ programs, but are
used by the implementation to do noninterference checking.

\subsection{Valid RPLs}
\label{section:rpls:valid}

A region path list (RPL) is a nonempty sequence of RPL elements
(\S~\ref{section:rpls:elts}) separated by colons (\kwd{:}).  An RPL
is well-formed if it obeys the following rules:
%
\begin{itemize}
%
\item The first element must be \kwd{Root}, an RPL parameter element
  (\S~\ref{section:rpls:elts:array}), or a variable RPL element
  (\S~\ref{section:rpls:elts:var}).
%
\item \kwd{Root}, an RPL parameter element, or a variable element may
  appear only as the first element.
%
\end{itemize}
%
It is also acceptable to write an RPL starting with a class or local
region name element (\S~\ref{section:rpls:elts:name-elts}), an array
index element (\S~\ref{section:rpls:elts:array}), or the star element
(\S~\ref{section:rpls:elts:star}).  In these cases, the RPL has
\kwd{Root} as an implicit first element.


\subsection{RPL Elements}
\label{section:rpls:elts}

An RPL element is one of the following: a name RPL element, a
parameter RPL element, an array index RPL element, a variable RPL
element, or the star RPL element.
%

\subsubsection{Name RPL Elements}
\label{section:rpls:elts:name-elts}

A name RPL element is one of the following:
%
\begin{itemize}
%
\item \kwd{Root}, which is a reserved name available at global scope.
%
\item A class region name declared in the enclosing class scope
  (\S~\ref{section:class-def:class-rgn-decl}).
%
\item A local region name declared in the enclosing statement scope
  (\S~\ref{section:class-def:methods:local-rgn-decl}).
%
\item \emph{selector}\kwd{.}\emph{region}, where \emph{selector} is a
  field access selector (i.e., everything to the left of
  \kwd{.}\emph{Identifier} in a field access expression, JLS \S~
  15.11), and \emph{region} is a class region declared in the class
  named by \emph{selector} (\S~\ref{section:class-def:class-rgn-decl}).
%
\end{itemize}
%

\subsubsection{Parameter RPL Elements}
\label{section:rpls:elts:param}

A parameter RPL element names a class RPL parameter
(\S~\ref{section:class-def:params}), method RPL parameter
(\S~\ref{section:class-def:params:method}), or capture parameter
(\S~\ref{section:types:class:capture}) that is in scope.  Note that
capture parameters are introduced by the compiler in capturing types
(\S~\ref{section:types:class:capture}), and cannot be written by the
programmer.

\subsubsection{Array Index RPL Elements}
\label{section:rpls:elts:array}

An array index RPL element is one of the following:
%
\begin{itemize}
%
\item \kwd{[}\emph{expr}\kwd{]}, where \emph{expr} is a valid integer
  expression, indicating the array index corresponding to the runtime
  value of \emph{expr}.
%
\item \kwd{[?]}, indicating any array index.
%
\end{itemize}

\subsubsection{Variable RPL Elements}
\label{section:rpls:elts:var}

A variable RPL element is one of the following: \kwd{this}; a
\kwd{final} method variable of class type; or a \kwd{final} method
formal parameter of class type.

\subsubsection{The Star RPL Element}
\label{section:rpls:elts:star}

The star RPL element has the form \kwd{*}.  It indicates any sequence
of RPL elements.

\subsection{Relations on RPLs}
\label{section:rpls:relations}

RPLs have several pairwise relations that are used to determine when
two RPLs name disjoint or overlapping sets of regions: equivalence
(\S~\ref{section:rpls:relations:equiv}), nesting
(\S~\ref{section:rpls:relations:nesting}), disjointness
(\S~\ref{section:rpls:relations:disjoint}), and inclusion
(\S~\ref{section:rpls:relations:inclusion}).

\subsubsection{Equivalence}
\label{section:rpls:relations:equiv}

There is an equivalence relation on RPLs, which we will denote here by
``\emph{rpl-1} is equivalent to \emph{rpl-2}.''  Its meaning is that
the set of regions represented by \emph{rpl-1} is equal to the set of
regions represented by \emph{rpl-2}.

\noindent
\textbf{Equivalent RPLs:} Two RPLs are equivalent if both RPLs have
the same number of elements, and the corresponding pairs of elements
are equivalent.  

\noindent
\textbf{Equivalent RPL Elements:} Two RPL elements are equivalent if:
%
\begin{itemize}
%
\item They refer to the declared same region name, RPL parameter, or
  variable name; or
%
\item They are \kwd{[}\emph{expr-1}\kwd{]} and
  \kwd{[}\emph{expr-2}\kwd{]}, where \emph{expr-1} and \emph{expr-2}
  are always-equal expressions; or
%
\item They are both \kwd{[?]} or \kwd{*}.
%
\end{itemize}

\noindent
\textbf{Always-Equal Expressions:} Always-equal expressions are
defined as follows:
%
\begin{itemize}
%
\item Two constants are always-equal expressions if they are the same
  value.
%
\item Two variables are always-equal expressions if they are the same
  variable (i.e., they have the same variable symbol) and the variable
  is declared \kwd{final}.
%
\item Two binary operation expressions are always-equal expressions if
  they represent the same operation, and their corresponding component
  expressions are always-equal.
%
\item Any other two expressions are not always-equal.
%
\end{itemize}


\subsubsection{Nesting}
\label{section:rpls:relations:nesting}

There is a nesting relation on RPLs, which we will denote here by
``\emph{rpl-1} is nested under \emph{rpl-2}.''  Its meaning is that
for every region \emph{r-1} represented by \emph{rpl-1}, there exists
a region \emph{r-2} represented by \emph{rpl-2} such that \emph{r-1}
is nested under \emph{r-2}.  The meaning of ``\emph{r-1} is nested
under \emph{r-2}'' Is that \emph{r-1} is a descendant of \emph{r-2} in
the region tree.
%
\emph{rpl-1} is nested under \emph{rpl-2} if one of the following holds:
%
\begin{itemize}
%
\item \emph{rpl-2} is \kwd{Root}.
%
\item \emph{rpl-1} is a sequence of two or more RPL elements, and
  \emph{rpl-1} without its last element is nested under \emph{rpl-2}.
%
\item \emph{rpl-1} is included in \emph{rpl-2}
  (\S~\ref{section:rpls:relations:inclusion}).
%
\item \emph{rpl-1} has one element, which is an object reference
  variable whose owner RPL (\S~\ref{section:types:class:owner-rpls}) is
  nested under \emph{rpl-2}.
%
\end{itemize}


\subsubsection{Inclusion}
\label{section:rpls:relations:inclusion}

There is an inclusion relation on RPLs, which we will denote here by
``\emph{rpl-1} is included in \emph{rpl-2}.''  Its meaning is that the
set of regions represented by \emph{rpl-1} is contained (in the sense
of ordinary set inclusion) in the set of regions represented by
\emph{rpl-2}.

\noindent
\textbf{Inclusion of RPLs:} \emph{rpl-1} is included in \emph{rpl-2}
if one of the following holds:
%
\begin{itemize}
%
\item \emph{rpl-1} and \emph{rpl-2} are equivalent
  (\S~\ref{section:rpls:relations:equiv}).
%
\item \emph{rpl-2} ends with \kwd{*}, and \emph{rpl-1} is nested under
  \emph{rpl-2} (\S~\ref{section:rpls:relations:nesting}) without its last element.
%
\item The last element of \emph{rpl-1} is included in the last element
  of \emph{rpl-2}, and inclusion holds for the RPLs after stripping
  the last elements of both.
%
\item \emph{rpl-1} has one element, which is a capture parameter
  (\S~\ref{section:types:class:capture}) included in \emph{rpl-2}.
%
\end{itemize}
%

\noindent
\textbf{Inclusion of RPL Elements:} RPL element \emph{elt-1} is
included in RPL element \emph{elt-2} if (1) the elements are
equivalent (\S~\ref{section:rpls:relations:equiv}); or (2) \emph{elt-1}
is any array index RPL element and \emph{elt-2} is \kwd{[?]}.


\subsubsection{Disjointness}
\label{section:rpls:relations:disjoint}

There is a disjointness relation on RPLs, which we will denote here by
``\emph{rpl-1} is disjoint from \emph{rpl-2}.''  Its meaning is that
the set of regions represented by \emph{rpl-1} has empty intersection
with the set of regions represented by \emph{rpl-2}.

\noindent
\textbf{Disjoint RPLs:} \emph{rpl-1} and \emph{rpl-2} are disjoint if
one of the following holds:
%
\begin{enumerate}
%
\item \emph{rpl-1} is included in
  (\S~\ref{section:rpls:relations:inclusion}) some RPL \emph{rpl-3};
  \emph{rpl-2} is included in some RPL \emph{rpl-4}; and \emph{rpl-3}
  and \emph{rpl-4} are constrained to be disjoint
  (\S~\ref{section:class-def:params}).
%
\item For some $n \geq 1$, (a) the RPLs formed by taking the first $n$
  elements of \emph{rpl-1} and \emph{rpl-2} are equivalent
  (\S~\ref{section:rpls:relations:equiv}); and (b) \emph{rpl-1} and
  \emph{rpl-2} have disjoint elements in position $n+1$ (this is
  called a ``distinction from the left'' in~\cite{DPJ:OOPSLA09}).
%
\item The last elements of \emph{rpl-1} and \emph{rpl-2} are disjoint
  (this is called a ``distinction from the right''
  in~\cite{DPJ:OOPSLA09}).
%
\end{enumerate}

\noindent
\textbf{Disjoint RPL Elements:} Two RPL elements \emph{elt-1} and
\emph{elt-2} are disjoint if (1) one is a name RPL element
(\S~\ref{section:rpls:elts:name-elts}) and the other is an array index
RPL element (\S~\ref{section:rpls:elts:array}); or (2) they are both
name RPL elements corresponding to distinct region name symbols
(\S\S~\ref{section:class-def:class-rgn-decl}
and~\ref{section:class-def:methods:local-rgn-decl}); or (3) they are
the array RPL elements \kwd{[}\emph{expr-1}\kwd{]} and
\kwd{[}\emph{expr-2}\kwd{]}, where \emph{expr-1} and \emph{expr-2} are
always-unequal expressions.


\noindent
\textbf{Always-Unequal Expressions:} Always-unequal expressions are
defined as follows:
%
\begin{itemize}
%
\item Two constants are always-unequal expressions if they are
  different values.
%
\item \emph{expr-1} and \emph{expr-2} are always-unequal expressions
  if \emph{expr-1} is the negation expression
  (\S~\ref{section:parallel:foreach}) of \emph{expr-2}, or vice versa.
%
% This isn't implemented
%
%\item Two add or subtract expressions are always-unequal expressions
%  if they represent the same operation and (1) their LHS expressions
%  are always-unequal, and their RHS expressions are always-equal; or
%  (2) vice versa.
%
\item Any other two expressions are not always-unequal.
%
\end{itemize}

\subsection{Regions}
\label{section:rpls:regions}

DPJ partitions memory into \emph{heap regions} (collections of memory
locations on the heap) and \emph{stack regions} (collections of memory
locations on the stack).  Heap regions are explicitly named using RPLs
(\S~\ref{section:rpls}).  Stack regions are internally named and
automatically managed by the compiler.

Because all DPJ type and effect annotations are erased at compile
time, regions are not actually represented at runtime.  They are a
theoretical construct, used in the soundness proofs of
DPJ~\cite{DPJ:FormalTR08}.  For example, regions are used to show that
if two RPLs are disjoint (\S~\ref{section:rpls:relations:disjoint}),
then the two sets of memory locations associated with the RPLs in the
dynamic semantics are nonintersecting.

The programmer should not have to reason directly about regions; the
proofs in~\cite{DPJ:FormalTR08} essentially say that it is sufficient
to reason about RPLs, using the rules given in \S~\ref{section:rpls}.
For more information about regions and the region tree, the interested
reader is referred to~\cite{DPJ:FormalTR08}.


\subsubsection{Heap Regions}
\label{section:regions:heap}

A heap region names a set of memory locations on the heap at runtime.
Syntactically, a heap region is like an RPL (\S~\ref{section:rpls}),
with the following exceptions:
%
\begin{itemize}
%
\item No \kwd{*} or \kwd{[?]} appears in the sequence of elements.
%
\item There are no parameter RPL elements.
%
\item Object reference values appear instead of object reference variables.
%
\item Integer values \kwd{[$n$]} appear instead of integer expressions
  \kwd{[}\emph{expr}\kwd{]}.
%
\end{itemize}
%
An RPL with no \kwd{*} or \kwd{[?]} corresponds to a heap region at
runtime, by replacing parameters with actual regions, object reference
variables with the references they store, and integer expressions with
the values to which they evaluate.  For more information about how
this substitution works, see~\cite{DPJ:FormalTR08}.  An RPL containing
\kwd{*} or \kwd{[?]}  represents the set of valid RPLs obtained by
substituting (possibly empty) sequences of RPL elements for \kwd{*}
and nonnegative array index values for \kwd{?}.  This set of RPLs in
turn corresponds to a set of regions, via the substitutions mentioned
above.

Every heap region is part of a tree of regions rooted at the global
name \kwd{Root} (\S~\ref{section:rpls:elts:name-elts}).  The tree
structure is given in two ways:
%
\begin{enumerate}
%
\item By the syntax of RPLs: e.g., \kwd{A:B} is a child of \kwd{A}.
%
\item By the types of object references: if $o$ is an object
  reference, and the owner region of the type of $o$
  (\S~\ref{section:types:class:owner-rpls}) is $R$, then a region whose
  RPL starts with $o$ is a descendant of region $R$.
%
\end{enumerate}
%
For more information about the region tree, see~\cite{DPJ:FormalTR08}.


\subsubsection{Stack Regions}
\label{section:regions:stack}

The compiler internally generates \emph{stack regions} for method
value parameters and local variables.  For example, the declaration
\kwd{int x} inside a method generates the stack region \kwd{x} that is
in scope where the variable \kwd{x} is in scope.  These stack regions
are generated and checked automatically, and the programmer cannot
specify them directly.

Stack regions are compared by symbol: two occurrences of the same
symbol are treated as identical, and different symbols are treated as
different.  This simple comparison is possible for stack variables
because Java does not allow references to or aliasing of these
variables.

