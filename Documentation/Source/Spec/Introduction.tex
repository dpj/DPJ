\section{Introduction}
\label{section:intro}

This document defines Deterministic Parallel Java (DPJ) as an
extension to the Java Programming Language, v1.6 (Java 6).  For the
specification of Java 6, see \emph{The Java Programming Language
  Specification}~\cite{JLS}, hereafter referred to as ``JLS.''
Because we are extending an already-specified language (Java), we omit
many of the topics one would expect to find in a specification
(lexical structure, form of statements and expressions, etc.).  We use
JLS for all that.  Instead, we describe only what DPJ adds to Java.

The structure of the specification is as follows.  In
\S~\ref{section:class-def}, we describe the DPJ extensions to Java's
class and interface definitions.  In \S~\ref{section:rpls}, we
describe \emph{region path lists}, or RPLs, which provide a way to
name sets of memory locations on the heap in a DPJ program.  In
\S~\ref{section:types}, we describe the DPJ extensions to Java's class
and array types.  In \S~\ref{section:effects} we describe
\emph{effects}, which specify accesses to memory in terms of
operations on RPLs.  In \S~\ref{section:parallel}, we describe DPJ's
constructs for expressing parallelism.

Readers of this document may wish to consult the following documents:
%
\begin{itemize}
%
\item \tutorial\ provides a guided introduction to the DPJ language.
%
\item \installmanual\ explains how to install DPJ on your system, and
  how to compile and run DPJ programs.
%
\item \refmanual\ explains the major features of the DPJ language in
  detail.  It contains many of the same concepts as this
  specification, but with more discussion, motivation, and examples.
  
\end{itemize}
%

Please note that the purpose of this specification is to give a
compact and precise \emph{definition} of DPJ, for someone who is
already familiar with the concepts discussed herein.  Therefore, it is
probably best to have at least a working knowledge of the tutorial and
reference manual (which are more concerned with \emph{explaining} the
language) before consulting this specification.
