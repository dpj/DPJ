\section{User Install%
\label{sec:user-install}}

This section explains how do a binary installation of DPJ, including
the bytecode version of the DPJ compiler and runtime and the source
code for the DPJ benchmarks.  If you install this way, then you can
compile and run DPJ programs.  However, you will not have access to
the source code for the DPJ compiler.  (The runtime source code is
included in the user install, because you need that to compile against
the runtime.  See \S~\ref{sec:invoking}.)  If you want to study or
modify the compiler and runtime source code, then you should install
from the \kwd{git} source base, as described in
\S~\ref{sec:dev-install}.

\subsection{Installation and setup}
To install the bytecode version of the compiler and runtime, do the
following:
%
\begin{enumerate}
%
\item Get the DPJ binary install tarball from
  \kwd{http://dpj.cs.uiuc.edu}.
%
\item Unpack the tarball:
%
\begin{verbatim}
tar -xvf dpjbin.tar 
\end{verbatim}
%
\item Set \kwd{DPJ\_ROOT} and your \kwd{PATH}:
%
\begin{verbatim}
setenv DPJ_ROOT ${HOME}/dpjbin
setenv PATH ${PATH}:${DPJ_ROOT}/Implementation/bin
\end{verbatim}
%
This assumes that \kwd{dpjbin} is in your home directory; if not, make
the necessary changes.
%
\item To check that you have a good installation, build the programs
  in the directory \kwd{Benchmarks/Kernels}:
%
\begin{verbatim}
cd dpjbin/Benchmarks/Kernels
make
\end{verbatim}
%
\item Test the kernels:
\begin{verbatim}
make test-all
\end{verbatim}

\end{enumerate}
%
You are now ready to compile and run DPJ programs.

\subsection{Compiling DPJ Programs}
\label{sec:invoking}

Compiling DPJ programs is a two-step process.  First, you invoke the
DPJ compiler \kwd{dpjc} to translate DPJ source to plain Java source.
Then you use an ordinary Java compiler (such as \kwd{javac}) to
translate the Java source to bytecode that can be run on a Java
virtual machine.

The following subsections explain these compilation steps in detail.
Further, the directory \kwd{\$\{DPJ\_ROOT\}/Benchmarks} contains a
file \kwd{Makefile.common} that illustrates how to set up a build
environment for managing these steps, including automatic management
of the subdirectories used to hold the translated Java and class
files.  You can also include the \kwd{Makefile.common} in your own
makefiles, getting the benefit of this setup with almost no effort.
See \kwd{\$\{DPJ\_ROOT\}/Benchmarks/Kernels/Makefile} for an example.

\subsubsection{Translating DPJ Source to Java Source}

You invoke the DPJ compiler by saying \kwd{dpjc} on the command line,
followed by some DPJ files to compile.  The DPJ compiler translates
the DPJ source to Java source, so it's best to direct the output of
the DPJ compiler to a different directory, to avoid name collisions.
For example, the command sequence
%
\begin{description}
\item \kwd{mkdir java}
\item \kwd{dpjc -d java Foo.java Bar.java}
\end{description}
%
translates the DPJ files \kwd{Foo.java} and \kwd{Bar.java} into the
plain Java files \kwd{Foo.java} and \kwd{Bar.java} in the directory
\kwd{java}.

Since \kwd{dpjc} is based on \kwd{javac}, you can use all the
command-line options that \kwd{javac} supports.  In addition,
\kwd{dpjc} supports the following options:
%
\begin{itemize}
%
\item \kwd{-seq}: If this flag is present, then the compiler
  translates the DPJ sources into sequential Java code.  By default,
  the compiler generates parallel code, using the \kwd{ForkJoinTask}
  library to express the parallelism implied by DPJ's \kwd{cobegin}
  and \kwd{foreach} constructs.
%
\item \kwd{-instrument}: This option makes sense only with \kwd{-seq}.
  With this option turned on, the compiler adds instrumentation to the
  program, so that when run its performance characteristics can be
  analyzed.  The instrumentation consists of method calls into the API
  defined by the \kwd{Instrument} class in the package
  \kwd{DPJRuntime}, located in \kwd{Implementation/Runtime}.  See the
  DPJ runtime API documentation for further details.
%
\item \kwd{-count}: When compiling the program, count the various
  kinds of DPJ annotations, and report the counts.
%
\end{itemize}

An important limitation of the DPJ compiler, as of DPJ v1.0, is that
it has limited support for separate compilation.  For example, in
ordinary Java, if \kwd{Foo.java} defines a class \kwd{Foo}, and
\kwd{Bar.java} defines a class \kwd{Bar} that uses \kwd{Foo}, you can
compile \kwd{Foo.java} to \kwd{Foo.class} separately, and later
compile \kwd{Bar.java} to \kwd{Bar.class}, as long as \kwd{Foo.class}
is in the class path specified on the compiler command line.

In DPJ, you can still do this if \kwd{Foo.java} is an ordinary Java
file (i.e., it doesn't have any DPJ annotations).  However, \emph{any
  source files containing DPJ annotations must be compiled together
  with code that depends on them}.  That is because DPJ's region and
effect annotations are currently not represented in the bytecode
(i.e., DPJ uses ordinary Java bytecode).  Therefore, the compiler
needs all the source files containing the DPJ annotations.  In
particular, in the example given above, if class \kwd{Foo} is defined
with a region parameter, and you attempt to compile class \kwd{Bar}
that uses \kwd{Foo} by linking against \kwd{Foo.class}, then the
compiler will generate an error, saying that \kwd{Foo} doesn't take
parameters.  That's because the region parameter information is
\emph{erased} in the \kwd{Foo.class} bytecode.  Instead, you need to
compile \kwd{Foo.java} and \kwd{Bar.java} together, so the compiler
can see the parameter.  The same limitation applies to the classes in
the DPJ runtime; see \S~7 of \refmanual\ for more details.

\subsubsection{Compiling Generated Java Source}

You can use any Java compiler to translate the generated Java source
to bytecode.  When you do this, you must put the DPJ runtime classes
in the class path.  For example, the command sequence
%
\begin{description}
\item \kwd{mkdir classes}
\item \kwd{javac -cp \$\{DPJ\_ROOT\}/Implementation/Runtime/classes -d
  classes java/*.java}
\end{description}
%
compiles the translated Java files in directory \kwd{java} to
bytecode, and puts the resulting class files in \kwd{classes}.  The
DPJ tools include a command \kwd{dpj-javac}, which is a convenience
wrapper for \kwd{javac} that includes the runtime classes in the class
path for you automatically.  For example
%
\begin{description}
\item \kwd{mkdir classes}
\item \kwd{dpj-javac -d classes java/*.java}
\end{description}
%
is equivalent to the above.  


\subsection{Running DPJ Programs}
\label{sec:running}

You run DPJ programs by saying \kwd{dpj} on the command line, followed
by one or more Java classes to execute.  The classes, and any classes
they depend on, must be in the class path.  For example:
%
\begin{description}
\item \kwd{dpj -cp classes Foo}
\end{description}
%
The options for setting the class path and the other runtime options
are the same as for \kwd{java}.  In fact, \kwd{dpj} just invokes the
ordinary \kwd{java} after adding the DPJ runtime classes to the class
path, so you can use \kwd{java} to run DPJ programs if you want; you
just have to add the runtime classes manually.  For example:
%
\begin{description}
\item \kwd{java -cp
  \$\{DPJ\_ROOT\}/Implementation/Runtime/classes:classes Foo}
\end{description}

The DPJ runtime has several configurable parameters.  These are set in
one of two ways.  First, they can be passed as command-line arguments
to the program.  The special DPJ command-line arguments must come
first; they are processed by the runtime and then stripped from the
argument list, which is passed to the main program.  For example, the
following invocation of the class \kwd{Foo} sets the DPJ foreach
cutoff (explained below) to 100, then passes \kwd{42} and \kwd{bar} as
the command-line arguments to the program:
%
\begin{description}
\item \kwd{dpj -cp classes Foo --dpj-foreach-cutoff 100 42 bar}
\end{description}

The following command-line options are processed specially by the
DPJ runtime as stated above.  Each of them must be followed by a
numeric argument; if not, an error is reported.  If the options are
not present, then the default is used as stated below.
%
\begin{itemize}
\item \kwd{--dpj-foreach-split} $n$: Set the branching factor used to
  split a \kwd{foreach} loop to $n$.  The loop is recursively split
  into this many branches in each iteration, until the cutoff is
  reached (see below).  The default is 2.
%
\item \kwd{--dpj-foreach-cutoff} $n$: Set the minimum number of
  \kwd{foreach} iterations allocated to a single task to $n$.  Beyond
  this point, no more parallel splitting of a foreach loop occurs.
  The default is 128.
%
\item \kwd{--dpj-num-threads} $n$: Set the number of worker threads
  used to run the program to $n$.  The default is the number of
  available processors given to the java virtual machine.
%
\end{itemize}

The second way to set the runtime parameters is to assign to them from
within the program.  This is useful if you want different
\kwd{foreach} loops in your program to use different parameters.  See
the runtime API documentation (available on the DPJ web site) for
information about how to do this.
