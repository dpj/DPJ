\documentclass[10pt]{article}

\usepackage{fullpage}
\usepackage{enumerate}
\usepackage{fouriernc}
\usepackage{color}
\usepackage{cite}
\usepackage{listings}
\usepackage{verbatim}
\usepackage{hevea}
\usepackage[pdfborder={0 0 0}]{hyperref}

\newcommand{\mb}[1]{\ensuremath{\mbox{#1}}}
\newcommand{\bnf}{\ensuremath{::=}}
\newcommand{\kwd}[1]{\mb{\small{\texttt{#1}}}}
\newcommand{\bfhead}[1]{\noindent\textbf{#1:}}
\newcommand{\ithead}[1]{\noindent\textit{#1:}}
\newcommand{\nonterm}[1]{\mb{\textit{#1}}}

\newcommand{\todo}[0]{\textbf{\textit{TODO}}}

\newcommand{\spec}{\emph{The Deterministic Parallel Java Language Specification}}
\newcommand{\tutorial}{\emph{The Deterministic Parallel Java Tutorial}}
\newcommand{\installmanual}{\emph{The Deterministic Parallel Java Installation Manual}}
\newcommand{\refmanual}{\emph{The Deterministic Parallel Java Language Reference Manual}}

\newcommand{\descr}[1]{\begin{description}\item #1\end{description}}

\lstdefinestyle{display}{
  emph={region, reads, writes, spawn, cobegin, under, foreach,
  finish}, emphstyle=\textbf, morecomment=[l]{//},
  basicstyle=\scriptsize, numbers=left }
\lstset{style=display}


\tocnumber
\setcounter{cuttingdepth}{0}


% Don't insert colored bars in table of contents
\newcommand{\newftocstyle}[3][0ex]{\newstyle{.ftoc#2}
  {list-style:none;margin:#1 1ex;padding:0ex 1ex;border-left:1ex solid transparent;}}  
  
% Orange section headings
%HEVEA\input{fancysection.hva}
%HEVEA\colorsections{26}

% Light green background for verbatim environments
%\newstyle{.verbatim}{margin:1ex 1ex;padding:1ex;background:\#ccffcc;}
\newstyle{.verbatim}{margin:1ex 1ex;padding:1ex;}

% Formatting for DPJ source code listings (with and without line numbers)
\lstdefinelanguage[DPJ]{Java} []{Java}
  {morendkeywords={in}, morendkeywords={region}, morendkeywords={type}, 
    morendkeywords={effect}, morendkeywords={reads}, morendkeywords={writes},
    morendkeywords={pure}, morendkeywords={commutative}, 
    morendkeywords={cobegin}, morendkeywords={foreach}}
\lstdefinestyle{listingstyle}{basicstyle=\normalsize\ttfamily,
  keywordstyle={\bf\ttfamily}, ndkeywordstyle={\bf\ttfamily\color{blue}},
  commentstyle={\em\color{magenta}}}
\lstnewenvironment{dpjlisting}
  {\setenvclass{lstlisting}{dpjlisting}
    \lstset{language=[DPJ]Java, style=listingstyle, numbers=none, 
      basicstyle=\small\ttfamily, showstringspaces=false, tabsize=4}}
  {}
\lstnewenvironment{numbereddpjlisting}
  {\setenvclass{lstlisting}{numbereddpjlisting}
    \lstset{language=[DPJ]Java, style=listingstyle, numbers=left, 
      basicstyle=\small\ttfamily, numberstyle=\scriptsize, showstringspaces=false, tabsize=4}}
  {}
\newstyle{.dpjlisting}{font-family:monospace;white-space:pre;
margin:1ex 1ex;padding:1ex;}
\newstyle{.numbereddpjlisting}{font-family:monospace;white-space:pre;
margin:1ex 1ex;padding:1ex 1ex 1ex 0ex;}




%%--------------------------------------------------------------------------- 
%% Main Document
%%--------------------------------------------------------------------------- 

\title{\bfseries{The Deterministic Parallel Java Installation Manual \\
Version 1.0}}
%
\author{University of Illinois at Urbana-Champaign}
%
\date{Revised July 2010}


\begin{document}

\maketitle

\begin{sloppypar}
This document explains how to install Deterministic Parallel Java
(DPJ), so you can use it to compile and run programs.  It also
explains how to build DPJ from source code, in case you wish to study
and/or modify the compiler internals.  A working knowledge of Java is
required. Please also consult \tutorial\ and \refmanual\ for
further details about the language.

The instructions in this document assume that you are working in a
UNIX-like system (including Linux, Mac OS X, etc.).  Everything should
work in Windows, but we have not tested DPJ on Windows systems.  To
follow the instructions as stated here, you will have to install
Cygwin on your Windows system to get a UNIX-like interface and tool
set.

\tableofcontents\cutname{contents.html}

\pagebreak
\section{Before You Begin%
\label{sec:build}}
\cutname{Building.html}

Thank you for you interest in DPJ.  This section covers the
information you need to know before installing DPJ on your system.

\subsection{Requirements%
\label{sec:requirements}}
\cutname{requirements.html}

DPJ is based on the Java language and \kwd{javac} compiler from Sun
Microsystems. This manual assumes that the reader is familiar with
Java, including compiling and running Java programs.  You will need at
least a Java Development Environment (JDK) to build and run DPJ codes.  If you wish
to work on the compiler sources you will need a Java Development Kit
(JDK) to build the compiler from source.  We recommend the latest Sun
JDK from \kwd{java.sun.com}.

\subsection{Developer or User?}

Instructions and requirements are slightly different depending on
whether you want to use DPJ \emph{(user install)} or develop DPJ
\emph{(developer install)}.  A user install of DPJ includes the
bytecode versions of the compiler and runtime, documentation, and the
DPJ source code for the benchmarks.  A developer install of DPJ
includes everything from the user install, and adds the compiler and
runtime sources and build environment.

Performing a user install is easier, but there are two reasons to do a
developer install:
%
\begin{enumerate}
%
\item You are interested in reading or modifying the source code for
  the DPJ compiler and/or runtime; or
%
\item You want access to the latest updates to the DPJ compiler,
  without waiting for the next bytecode release.
\end{enumerate}


\subsubsection{Requirements for User Install%
\label{sec:runDPJ}}
%
For a user install of DPJ, do the following:
\begin{enumerate}
\item Make sure you have a Java Devlopment Kit (JDK) installed.  As of
  this writing we have tested with JDK 1.6.0\_20.
\item Follow the instructions in \S~\ref{sec:user-install} to do
  the installation and start using DPJ.
\end{enumerate}

\subsubsection{Requirements for Developer Install%
\label{sec:buildDPJ}}
For a developer install of DPJ, do the following:
\begin{enumerate}
\item Make sure you have a JDK (at least Java 6) installed.
\item Make sure that you have a working installation of Apache
  \kwd{ant}, the Java build tool.
\item Make sure you have a working installation of \kwd{git}, the
  version control system.
%
\begin{description}
\item  \kwd{http://git.com}.
\end{description}
\item Make sure you have a working \kwd{make}.
\item The Eclipse IDE is invaluable for studying and modifying large
  Java programs, so we recommend installing it.  See
%
\begin{description}
\item \kwd{http://www.eclipse.org}
\end{description}
\item Follow the instructions in \S~\ref{sec:dev-install} to do the
  installation.  Then read \S\S~\ref{sec:invoking}
  and~\ref{sec:running} to learn how to use DPJ.
\end{enumerate}

\subsection{What's in the Release}
The release directory structure contains the following directories:
\begin{itemize}
\item \kwd{Documentation }: Manuals and instructions for using and/or
  building DPJ.
\item \kwd{Implementation}: The DPJ compiler and runtime.
\item \kwd{Benchmarks}: Example code kernels and applications written
  for DPJ.
\end{itemize}

 \subsection{DPJ Resources% 
\label{sec:getDPJ}}
You should keep the following resources in mind as you work with DPJ:
\begin{enumerate}
\item The DPJ home page: \kwd{dpj.cs.illinois.edu}.
\item The DPJ public code repository:
  \kwd{http://github.com/dpj/DPJ.git}.
\item The DPJ development mailing list: \kwd{dpjdev@cs.uiuc.edu}.
  Joining the list allows you to follow major announcements, news, and
  technical discussions regarding DPJ.  To subscribe to the list,
  please visit
%
\begin{description}
\item \kwd{http://lists.cs.uiuc.edu/mailman/listinfo/dpjdev}.
\end{description}

\item DPJ documentation is located in the \kwd{Documentation}
  directory of the DPJ release and is also available on the DPJ web
  site at \kwd{http://dpj.cs.illinois.edu/DPJ/Download.html}.
\end{enumerate}

The DPJ development team appreciates your feedback and questions.
Please check the email list archives before submitting a question or
bug report.  We prefer using the list for submissions so that the
entire DPJ community can benefit from the growing knowledge base of
questions and answers.


\subsection{DPJ License}

The DPJ software is subject to the following licenses:
%
\begin{itemize}
%
\item The DPJ compiler is based on Sun's \kwd{javac} compiler and is
  covered by the GNU General Public License version 2.
%
\item The programs in the \kwd{Benchmarks/Applications} directory are
  based on codes written by various third parties, and are subject to
  their licenses.
%
\item The rest of the code is by the University of Illinois and is
  released under the University of Illinois/NCSA Open Source License.
%
\end{itemize}
%
See the file \kwd{LICENSE.TXT} in the top-level directory of the DPJ
software for further license information.



\section{User Install%
\label{sec:user-install}}

This section explains how do a binary installation of DPJ, including
the bytecode version of the DPJ compiler and runtime and the source
code for the DPJ benchmarks.  If you install this way, then you can
compile and run DPJ programs.  However, you will not have access to
the source code for the DPJ compiler.  (The runtime source code is
included in the user install, because you need that to compile against
the runtime.  See \S~\ref{sec:invoking}.)  If you want to study or
modify the compiler and runtime source code, then you should install
from the \kwd{git} source base, as described in
\S~\ref{sec:dev-install}.

\subsection{Installation and setup}
To install the bytecode version of the compiler and runtime, do the
following:
%
\begin{enumerate}
%
\item Get the DPJ binary install tarball from
  \kwd{http://dpj.cs.uiuc.edu}.
%
\item Unpack the tarball:
%
\begin{verbatim}
tar -xvf dpjbin.tar 
\end{verbatim}
%
\item Set \kwd{DPJ\_ROOT} and your \kwd{PATH}:
%
\begin{verbatim}
setenv DPJ_ROOT ${HOME}/dpjbin
setenv PATH ${PATH}:${DPJ_ROOT}/Implementation/bin
\end{verbatim}
%
This assumes that \kwd{dpjbin} is in your home directory; if not, make
the necessary changes.
%
\item To check that you have a good installation, build the programs
  in the directory \kwd{Benchmarks/Kernels}:
%
\begin{verbatim}
cd dpjbin/Benchmarks/Kernels
make
\end{verbatim}
%
\item Test the kernels:
\begin{verbatim}
make test-all
\end{verbatim}

\end{enumerate}
%
You are now ready to compile and run DPJ programs.

\subsection{Compiling DPJ Programs}
\label{sec:invoking}

Compiling DPJ programs is a two-step process.  First, you invoke the
DPJ compiler \kwd{dpjc} to translate DPJ source to plain Java source.
Then you use an ordinary Java compiler (such as \kwd{javac}) to
translate the Java source to bytecode that can be run on a Java
virtual machine.

The following subsections explain these compilation steps in detail.
Further, the directory \kwd{\$\{DPJ\_ROOT\}/Benchmarks} contains a
file \kwd{Makefile.common} that illustrates how to set up a build
environment for managing these steps, including automatic management
of the subdirectories used to hold the translated Java and class
files.  You can also include the \kwd{Makefile.common} in your own
makefiles, getting the benefit of this setup with almost no effort.
See \kwd{\$\{DPJ\_ROOT\}/Benchmarks/Kernels/Makefile} for an example.

\subsubsection{Translating DPJ Source to Java Source}

You invoke the DPJ compiler by saying \kwd{dpjc} on the command line,
followed by some DPJ files to compile.  The DPJ compiler translates
the DPJ source to Java source, so it's best to direct the output of
the DPJ compiler to a different directory, to avoid name collisions.
For example, the command sequence
%
\begin{description}
\item \kwd{mkdir java}
\item \kwd{dpjc -d java Foo.java Bar.java}
\end{description}
%
translates the DPJ files \kwd{Foo.java} and \kwd{Bar.java} into the
plain Java files \kwd{Foo.java} and \kwd{Bar.java} in the directory
\kwd{java}.

Since \kwd{dpjc} is based on \kwd{javac}, you can use all the
command-line options that \kwd{javac} supports.  In addition,
\kwd{dpjc} supports the following options:
%
\begin{itemize}
%
\item \kwd{-seq}: If this flag is present, then the compiler
  translates the DPJ sources into sequential Java code.  By default,
  the compiler generates parallel code, using the \kwd{ForkJoinTask}
  library to express the parallelism implied by DPJ's \kwd{cobegin}
  and \kwd{foreach} constructs.
%
\item \kwd{-instrument}: This option makes sense only with \kwd{-seq}.
  With this option turned on, the compiler adds instrumentation to the
  program, so that when run its performance characteristics can be
  analyzed.  The instrumentation consists of method calls into the API
  defined by the \kwd{Instrument} class in the package
  \kwd{DPJRuntime}, located in \kwd{Implementation/Runtime}.  See the
  DPJ runtime API documentation for further details.
%
\item \kwd{-count}: When compiling the program, count the various
  kinds of DPJ annotations, and report the counts.
%
\end{itemize}

An important limitation of the DPJ compiler, as of DPJ v1.0, is that
it has limited support for separate compilation.  For example, in
ordinary Java, if \kwd{Foo.java} defines a class \kwd{Foo}, and
\kwd{Bar.java} defines a class \kwd{Bar} that uses \kwd{Foo}, you can
compile \kwd{Foo.java} to \kwd{Foo.class} separately, and later
compile \kwd{Bar.java} to \kwd{Bar.class}, as long as \kwd{Foo.class}
is in the class path specified on the compiler command line.

In DPJ, you can still do this if \kwd{Foo.java} is an ordinary Java
file (i.e., it doesn't have any DPJ annotations).  However, \emph{any
  source files containing DPJ annotations must be compiled together
  with code that depends on them}.  That is because DPJ's region and
effect annotations are currently not represented in the bytecode
(i.e., DPJ uses ordinary Java bytecode).  Therefore, the compiler
needs all the source files containing the DPJ annotations.  In
particular, in the example given above, if class \kwd{Foo} is defined
with a region parameter, and you attempt to compile class \kwd{Bar}
that uses \kwd{Foo} by linking against \kwd{Foo.class}, then the
compiler will generate an error, saying that \kwd{Foo} doesn't take
parameters.  That's because the region parameter information is
\emph{erased} in the \kwd{Foo.class} bytecode.  Instead, you need to
compile \kwd{Foo.java} and \kwd{Bar.java} together, so the compiler
can see the parameter.  The same limitation applies to the classes in
the DPJ runtime; see \S~7 of \refmanual\ for more details.

\subsubsection{Compiling Generated Java Source}

You can use any Java compiler to translate the generated Java source
to bytecode.  When you do this, you must put the DPJ runtime classes
in the class path.  For example, the command sequence
%
\begin{description}
\item \kwd{mkdir classes}
\item \kwd{javac -cp \$\{DPJ\_ROOT\}/Implementation/Runtime/classes -d
  classes java/*.java}
\end{description}
%
compiles the translated Java files in directory \kwd{java} to
bytecode, and puts the resulting class files in \kwd{classes}.  The
DPJ tools include a command \kwd{dpj-javac}, which is a convenience
wrapper for \kwd{javac} that includes the runtime classes in the class
path for you automatically.  For example
%
\begin{description}
\item \kwd{mkdir classes}
\item \kwd{dpj-javac -d classes java/*.java}
\end{description}
%
is equivalent to the above.  


\subsection{Running DPJ Programs}
\label{sec:running}

You run DPJ programs by saying \kwd{dpj} on the command line, followed
by one or more Java classes to execute.  The classes, and any classes
they depend on, must be in the class path.  For example:
%
\begin{description}
\item \kwd{dpj -cp classes Foo}
\end{description}
%
The options for setting the class path and the other runtime options
are the same as for \kwd{java}.  In fact, \kwd{dpj} just invokes the
ordinary \kwd{java} after adding the DPJ runtime classes to the class
path, so you can use \kwd{java} to run DPJ programs if you want; you
just have to add the runtime classes manually.  For example:
%
\begin{description}
\item \kwd{java -cp
  \$\{DPJ\_ROOT\}/Implementation/Runtime/classes:classes Foo}
\end{description}

The DPJ runtime has several configurable parameters.  These are set in
one of two ways.  First, they can be passed as command-line arguments
to the program.  The special DPJ command-line arguments must come
first; they are processed by the runtime and then stripped from the
argument list, which is passed to the main program.  For example, the
following invocation of the class \kwd{Foo} sets the DPJ foreach
cutoff (explained below) to 100, then passes \kwd{42} and \kwd{bar} as
the command-line arguments to the program:
%
\begin{description}
\item \kwd{dpj -cp classes Foo --dpj-foreach-cutoff 100 42 bar}
\end{description}

The following command-line options are processed specially by the
DPJ runtime as stated above.  Each of them must be followed by a
numeric argument; if not, an error is reported.  If the options are
not present, then the default is used as stated below.
%
\begin{itemize}
\item \kwd{--dpj-foreach-split} $n$: Set the branching factor used to
  split a \kwd{foreach} loop to $n$.  The loop is recursively split
  into this many branches in each iteration, until the cutoff is
  reached (see below).  The default is 2.
%
\item \kwd{--dpj-foreach-cutoff} $n$: Set the minimum number of
  \kwd{foreach} iterations allocated to a single task to $n$.  Beyond
  this point, no more parallel splitting of a foreach loop occurs.
  The default is 128.
%
\item \kwd{--dpj-num-threads} $n$: Set the number of worker threads
  used to run the program to $n$.  The default is the number of
  available processors given to the java virtual machine.
%
\end{itemize}

The second way to set the runtime parameters is to assign to them from
within the program.  This is useful if you want different
\kwd{foreach} loops in your program to use different parameters.  See
the runtime API documentation (available on the DPJ web site) for
information about how to do this.

\section{Developer Install%
\label{sec:dev-install}}

\subsection{Installation and setup}

To install the DPJ compiler and runtime from source, do the following.
%
\begin{enumerate}
%
\item Check that your system meets the requirements in
  \S~\ref{sec:requirements}.
%
\item Get the source code distribution.  We recommend using Eclipse to
  check out the source code; that way you'll have access to all of
  Eclipse's code browsing and editing features.  Go to
  \kwd{http://www.eclipse.org} to get Eclipse.  Then you'll need to install
  EGit (Git for Eclipse).  This tutorial tells you how to do it:
%
\begin{description}
\item \kwd{http://www.vogella.de/articles/EGit/article.html}
\end{description}
%
Then you can check out the DPJ source tree from inside Eclipse, again
as explained in the tutorial.  Use this as the repository URL:
%
\begin{description}
\item \kwd{git://github.com/dpj/DPJ.git}
\end{description}
%
The following steps assume you are calling the root of the working
directory \kwd{DPJ} and putting it in your home directory.  If not,
make the appropriate adjustments.

If you don't want to use Eclipse, you can check out the DPJ source
tree from the command line:
%
\begin{verbatim}
cd ~
git clone git://github.com/dpj/DPJ.git DPJ
\end{verbatim}
%
Again, make the appropriate adjustments if you are putting the root of
the working directory somewhere else.
%
\item Set the \kwd{DPJ\_ROOT} environment variable.  For example, if
  the root of your DPJ tree is \kwd{DPJ} in your home directory and
  you are using the C shell, put this in your \kwd{.cshrc} file:
%
\begin{verbatim}
setenv DPJ_ROOT ~/DPJ
\end{verbatim}
%
Then do
%
\begin{verbatim}
source ~/.cshrc
\end{verbatim}
%
\item Tell \kwd{ant} to build the compiler, using your JDK: 
%
\begin{verbatim}
cd ~/DPJ/Implementation/Compiler/make
ant -Dboot.java.home=/usr/lib/jvm/java
\end{verbatim}
%
If your JDK is not located at \kwd{/usr/bin/jvm/java}, then substitute
the appropriate path.
%
\item A successful compiler build puts \kwd{jar} files for the
  compiler and related tools (such as \kwd{javadoc}) in
\begin{description}
\item \kwd{Implementation/Compiler/build}.  
\end{description}
The executable files \kwd{dpjc} and \kwd{dpj} in
\kwd{Implementation/bin} invoke these \kwd{jar} files.
%
\item Add the DPJ compiler bin to your \kwd{PATH}:
\begin{verbatim}
setenv PATH ${PATH}:${DPJ_ROOT}/Implementation/bin/
\end{verbatim}
\item Build the runtime classes located at
  \kwd{{\$\{DPJ\_ROOT\}}/Implementation/Runtime}.  You can expect some
  warnings.
\begin{verbatim}
cd ../../Runtime/
make
\end{verbatim}

\item You can now build and run the kernels:
\begin{verbatim}
cd ${DPJ_ROOT}/Benchmarks/Kernels
make
\end{verbatim}

\item Then run all the kernel tests: 
\begin{verbatim}
make test-all
\end{verbatim}


\end{enumerate}

\subsection{Testing the Installation%
\label{sec:testing}}
\begin{enumerate}
\item JUnit 4 tests are in
  \kwd{\$\{DPJ\_ROOT\}/Implementation/Compiler/test/dpj-junit-tests}.  The
  corresponding DPJ source files are in \kwd{test/dpj-programs}.
\item The JUnit tests are designed to be run from within Eclipse, by
  running the launcher file
%
\begin{description}
\item \kwd{test/dpj-junit-tests/dpj-junit-tests.launch}.
\end{description}
%
If the JUnit tests fail with a \kwd{MethodNotFound} error, it is
probably because the classpath is wrong.  Many of the javac classes
are included in the system library (\kwd{rt.jar}), and so the DPJ
versions must precede the system library in the classpath.  If you use
the launcher, this error should not happen.
\end{enumerate}

\subsection{Browsing and Modifying the Compiler Code}

After an initial \kwd{ant} build, DPJ javac will build automatically
in Eclipse (albeit to a \kwd{bin/} directory instead of \kwd{build/}).
However, if any of the resource bundles are changed (be sure to change
the \kwd{.properties} files in
%
\begin{description}
\item \kwd{src/share/classes/com/sun/tools/javac/resources} 
\end{description}
and not the generated \kwd{.java} files!), the compiler must be
re-built using the \kwd{ant} script.
 
The \kwd{ant} script bootstraps the compiler.  So if there is an
internal compiler error, it is due to a fault in the DPJ compiler.
Note also that the compiler may compile successfully in Eclipse but
fail when it is bootstrapped in the \kwd{ant} script due to differences
between pure Java and DPJ (or an error in the DPJ compiler).  In
particular, note that the following keywords are reserved in DPJ and
therefore may not be used as variable names (in any DPJ program,
including the DPJ compiler):
\begin{itemize}
     \item \kwd{under}
     \item \kwd{region}
     \item \kwd{reads}
     \item \kwd{writes}
\end{itemize}





\end{sloppypar}

\end{document}

