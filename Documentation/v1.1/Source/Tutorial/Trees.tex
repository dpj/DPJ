\section{Computations on Trees%
\label{sec:tree}}
\cutname{tree.html}

This section explains how to write two common parallel patterns for
tree processing in DPJ:
%
\begin{itemize}
%
\item \S~\ref{sec:tree:build} shows how to build a tree data structure
  in parallel, using a divide-and-conquer strategy.
%
\item \S~\ref{sec:tree:update} shows how to recurse over a tree in
  parallel and update its nodes.
%
\end{itemize}
%

\subsection{Recursive Tree Build%
\label{sec:tree:build}}

This section shows how to use DPJ to implement a divide-and-conquer
build of a tree data structure.  We show how to build the tree so it
can be used for a parallel algorithm that traverses and updates its
nodes.  This requires that every node have a different region in its
type, and the regions have a tree structure.  If read-only access to
the tree is desired, then the DPJ types can be simplified.  We leave
that simplification to the reader as an exercise.

\begin{figure}
\input{Listings/RecursiveTreeBuild}
\caption{Recursive Tree Build.}
\label{fig:tree:build}
\end{figure}

\bfhead{How to write the pattern} Figure~\ref{fig:tree:build}
illustrates the Recursive Tree Build pattern, for an algorithm that
builds a binary region tree for partitioning bodies in one-dimensional
space.  In this kind of tree, the data (the bodies) are in the leaf
nodes; the inner nodes are there just to facilitate searching the
data.  For example, computing whether a ray intersects a body can be
done in $\mbox{log}(n)$ time, where $n$ is the number of points.

Lines 6--20 of Figure~\ref{fig:tree:build} define the classes that the
algorithm uses:
%
\begin{itemize}
%
\item Class \kwd{Body} (lines 6--9) defines a body with a mass and a
  position.  The fields are declared \kwd{final}, so reading them has
  no effect.
%
\item Class \kwd{Node} (line 10) defines an abstract node of the tree.
  It contains a \kwd{Body} representing its center of mass.  For a
  leaf node, this is an actual physical body; and for an inner node,
  this is the average of all the bodies in the leaves under the node.
  There is one region parameter \kwd{R}, so that different instances
  of \kwd{Node} can have different regions, and the \kwd{centerOfMass}
  field is in region \kwd{R}.
%
\item Class \kwd{InnerNode} (lines 11--17) defines an inner node of
  the tree.  It inherits the \kwd{centerOfMass} field from \kwd{Node}.
  Additionally, it has a left bound and a right bound, defining a
  region of 1-D space; and it has a left child and a right child,
  defining the tree.  Here we have used the RPLs \kwd{R:LeftRgn} and
  \kwd{R:RightRgn} to give the region of the tree node a region
  structure that mirrors the tree structure.  For example, at runtime,
  the left child of the right child of the root node would have the
  type \kwd{Node<Root:RightRgn:LeftRgn>}.
%
\item Class \kwd{LeafNode} (lines 18--20) just adds a constructor to
  \kwd{Node}, so that a leaf node representing a physical body may be
  created.
\end{itemize}

The rest of the code defines the methods that do the tree building:
%
\begin{itemize}
%
\item Method \kwd{computeSplitPoint} (lines 21--30) is a helper method
  that takes an array of bodies and a point in space.  It computes and
  returns the index into the array such that the elements to the left
  of the index are all and only the elements to the left of the point
  in space.  The array is assumed to be sorted by position.  We don't
  show this code.  It can easily be implemented in parallel using a
  divide-and-conquer strategy.
%
\item Method \kwd{makeTree} (lines 31 and following) does the actual
  tree building and is discussed below.
%
\end{itemize}

Method \kwd{makeTree} uses divide-and-conquer recursion to build the
tree.  It takes an array of bodies to build into a tree (assumed to be
pre-sorted by position), and a left and right bound representing the
region of space that the tree is partitioning.  It returns a node
representing the root of the tree.  The method has two region
parameters, \kwd{RN} and \kwd{RA}.  \kwd{RN} is the region associated
with the type of the node returned, while \kwd{RA} is the region
associated with the array of bodies coming in.  The parameters are
constrained so that writes under \kwd{RN} and under \kwd{RA} are
noninterfering.  (For example two different names \kwd{A} and \kwd{B}
would satisfy this constraint, but \kwd{A} and \kwd{A:B} would not.)

The body of \kwd{makeTree} does the following:
%
\begin{itemize}
%
\item If the input array is empty, return \kwd{null} (line 36).
%
\item If the input array has just one element, then create a fresh
  leaf node for it (line 37).
%
\item Otherwise, handle the recursive case.
%
\end{itemize}

The recursive case works as follows:
%
\begin{itemize}
%
\item Compute the index into the array corresponding to the midpoint
  of the current region of space, and split the array there (lines
  38--40).
%
\item Make a new \kwd{InnerNode} corresponding to the current region
  of space (line 41).
%
\item In parallel, populate the left and right children of the new
  inner node, by calling \kwd{makeTree} recursively (lines 42--47).
%
\item Return the created node (line 48).
%
\end{itemize}

The interesting types and effects are inside the \kwd{cobegin}.
Notice that lines 43--44 call \kwd{makeTree} with
$\kwd{RN}=\kwd{RN:InnerNode.LeftRgn}$, which creates the right type
for assigning into \kwd{node.leftChild}.  Notice also that the effects
of the two branches of the \kwd{cobegin} are \kwd{reads RA:* writes
  RN:InnerNode.LeftRgn:*} and \kwd{reads RA:* writes
  RN:InnerNode.RightRgn:*}, and these two effects are disjoint.

\bfhead{Further examples} The file \kwd{Quadtree.java} in directory
\kwd{Benchmarks/Kernels/dpj} of the DPJ uses this pattern to build a
quadtree (i.e., a tree in which each node has up to four children) for
partitioning two-dimensional space.  The main difference is that
index-parameterized arrays (Reference Manual \S~4.3) are used instead
of the regions \kwd{LeftRgn} and \kwd{RightRgn} to distinguish the
different branches of the tree.  The partition step is implemented
sequentially, but it could easily be parallelized.

The Barnes-Hut benchmark in the DPJ release
(\kwd{Benchmarks/Applications/Barnes-Hut/dpj}) builds an octree for
partitioning three-dimensional space.  The tree build phase is not
parallelized in that benchmark, but it could be using this pattern.

\subsection{Recursive Tree Update%
\label{sec:tree:update}}

This section shows how to traverse the tree created in
\S~\ref{sec:tree:build} in parallel and update its nodes.  The tree
structure of the regions in the types of the nodes allow this
traversal and update to be done with provable noninterference.

\begin{figure}
\input{Listings/RecursiveTreeUpdate}
\caption{Recursive Tree Update.}
\label{fig:tree:update}
\end{figure}

\bfhead{How to write the pattern} Figure~\ref{fig:tree:update}
illustrates the Recursive Tree Update pattern, for an algorithm that
traverses the tree constructed in Figure~\ref{fig:tree:build} and
updates the center of mass of each node.  The work is done by a single
method, \kwd{updateCenterOfMass}, which accepts a tree node, updates
the center of mass fields of that node and all its descendants, and
returns the center of mass of that node.

The method body works as follows.  If the input node is a leaf node,
then the method just returns its body as the center of mass (lines
10--11).  Otherwise, the method recursively and in parallel calls
\kwd{updateCenterOfMass} on the left and right nodes (lines 13--20)
and constructs a new \kwd{Body} representing the center of mass (lines
21--28).  It then stores the \kwd{Body} representing the center of
mass into the input node and returns the \kwd{Body}.

In lines 15--20, the effects of each of the two branches of the
\kwd{cobegin} statement are as shown.  Because of the tree structure
of the regions of the tree nodes, the effects on the left and write
subtrees are disjoint.

\bfhead{Further examples} The Barnes-Hut benchmark in the DPJ release
(\kwd{Benchmarks/Applications/Barnes-Hut/dpj}) does a similar
center-of-mass computation on an octtree.  The center-of-mass phase is
not parallelized in that benchmark, but it could be using this
pattern.


